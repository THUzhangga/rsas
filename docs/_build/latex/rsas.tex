% Generated by Sphinx.
\def\sphinxdocclass{report}
\documentclass[letterpaper,10pt,english]{sphinxmanual}
\usepackage[utf8]{inputenc}
\DeclareUnicodeCharacter{00A0}{\nobreakspace}
\usepackage{cmap}
\usepackage[T1]{fontenc}
\usepackage{babel}
\usepackage{times}
\usepackage[Bjarne]{fncychap}
\usepackage{longtable}
\usepackage{sphinx}
\usepackage{multirow}



\title{rsas Documentation}
\date{January 17, 2015}
\release{0.1.1}
\author{Ciaran J. Harman}
\newcommand{\sphinxlogo}{}
\renewcommand{\releasename}{Release}
\makeindex

\makeatletter
\def\PYG@reset{\let\PYG@it=\relax \let\PYG@bf=\relax%
    \let\PYG@ul=\relax \let\PYG@tc=\relax%
    \let\PYG@bc=\relax \let\PYG@ff=\relax}
\def\PYG@tok#1{\csname PYG@tok@#1\endcsname}
\def\PYG@toks#1+{\ifx\relax#1\empty\else%
    \PYG@tok{#1}\expandafter\PYG@toks\fi}
\def\PYG@do#1{\PYG@bc{\PYG@tc{\PYG@ul{%
    \PYG@it{\PYG@bf{\PYG@ff{#1}}}}}}}
\def\PYG#1#2{\PYG@reset\PYG@toks#1+\relax+\PYG@do{#2}}

\expandafter\def\csname PYG@tok@gd\endcsname{\def\PYG@tc##1{\textcolor[rgb]{0.63,0.00,0.00}{##1}}}
\expandafter\def\csname PYG@tok@gu\endcsname{\let\PYG@bf=\textbf\def\PYG@tc##1{\textcolor[rgb]{0.50,0.00,0.50}{##1}}}
\expandafter\def\csname PYG@tok@gt\endcsname{\def\PYG@tc##1{\textcolor[rgb]{0.00,0.27,0.87}{##1}}}
\expandafter\def\csname PYG@tok@gs\endcsname{\let\PYG@bf=\textbf}
\expandafter\def\csname PYG@tok@gr\endcsname{\def\PYG@tc##1{\textcolor[rgb]{1.00,0.00,0.00}{##1}}}
\expandafter\def\csname PYG@tok@cm\endcsname{\let\PYG@it=\textit\def\PYG@tc##1{\textcolor[rgb]{0.25,0.50,0.56}{##1}}}
\expandafter\def\csname PYG@tok@vg\endcsname{\def\PYG@tc##1{\textcolor[rgb]{0.73,0.38,0.84}{##1}}}
\expandafter\def\csname PYG@tok@m\endcsname{\def\PYG@tc##1{\textcolor[rgb]{0.13,0.50,0.31}{##1}}}
\expandafter\def\csname PYG@tok@mh\endcsname{\def\PYG@tc##1{\textcolor[rgb]{0.13,0.50,0.31}{##1}}}
\expandafter\def\csname PYG@tok@cs\endcsname{\def\PYG@tc##1{\textcolor[rgb]{0.25,0.50,0.56}{##1}}\def\PYG@bc##1{\setlength{\fboxsep}{0pt}\colorbox[rgb]{1.00,0.94,0.94}{\strut ##1}}}
\expandafter\def\csname PYG@tok@ge\endcsname{\let\PYG@it=\textit}
\expandafter\def\csname PYG@tok@vc\endcsname{\def\PYG@tc##1{\textcolor[rgb]{0.73,0.38,0.84}{##1}}}
\expandafter\def\csname PYG@tok@il\endcsname{\def\PYG@tc##1{\textcolor[rgb]{0.13,0.50,0.31}{##1}}}
\expandafter\def\csname PYG@tok@go\endcsname{\def\PYG@tc##1{\textcolor[rgb]{0.20,0.20,0.20}{##1}}}
\expandafter\def\csname PYG@tok@cp\endcsname{\def\PYG@tc##1{\textcolor[rgb]{0.00,0.44,0.13}{##1}}}
\expandafter\def\csname PYG@tok@gi\endcsname{\def\PYG@tc##1{\textcolor[rgb]{0.00,0.63,0.00}{##1}}}
\expandafter\def\csname PYG@tok@gh\endcsname{\let\PYG@bf=\textbf\def\PYG@tc##1{\textcolor[rgb]{0.00,0.00,0.50}{##1}}}
\expandafter\def\csname PYG@tok@ni\endcsname{\let\PYG@bf=\textbf\def\PYG@tc##1{\textcolor[rgb]{0.84,0.33,0.22}{##1}}}
\expandafter\def\csname PYG@tok@nl\endcsname{\let\PYG@bf=\textbf\def\PYG@tc##1{\textcolor[rgb]{0.00,0.13,0.44}{##1}}}
\expandafter\def\csname PYG@tok@nn\endcsname{\let\PYG@bf=\textbf\def\PYG@tc##1{\textcolor[rgb]{0.05,0.52,0.71}{##1}}}
\expandafter\def\csname PYG@tok@no\endcsname{\def\PYG@tc##1{\textcolor[rgb]{0.38,0.68,0.84}{##1}}}
\expandafter\def\csname PYG@tok@na\endcsname{\def\PYG@tc##1{\textcolor[rgb]{0.25,0.44,0.63}{##1}}}
\expandafter\def\csname PYG@tok@nb\endcsname{\def\PYG@tc##1{\textcolor[rgb]{0.00,0.44,0.13}{##1}}}
\expandafter\def\csname PYG@tok@nc\endcsname{\let\PYG@bf=\textbf\def\PYG@tc##1{\textcolor[rgb]{0.05,0.52,0.71}{##1}}}
\expandafter\def\csname PYG@tok@nd\endcsname{\let\PYG@bf=\textbf\def\PYG@tc##1{\textcolor[rgb]{0.33,0.33,0.33}{##1}}}
\expandafter\def\csname PYG@tok@ne\endcsname{\def\PYG@tc##1{\textcolor[rgb]{0.00,0.44,0.13}{##1}}}
\expandafter\def\csname PYG@tok@nf\endcsname{\def\PYG@tc##1{\textcolor[rgb]{0.02,0.16,0.49}{##1}}}
\expandafter\def\csname PYG@tok@si\endcsname{\let\PYG@it=\textit\def\PYG@tc##1{\textcolor[rgb]{0.44,0.63,0.82}{##1}}}
\expandafter\def\csname PYG@tok@s2\endcsname{\def\PYG@tc##1{\textcolor[rgb]{0.25,0.44,0.63}{##1}}}
\expandafter\def\csname PYG@tok@vi\endcsname{\def\PYG@tc##1{\textcolor[rgb]{0.73,0.38,0.84}{##1}}}
\expandafter\def\csname PYG@tok@nt\endcsname{\let\PYG@bf=\textbf\def\PYG@tc##1{\textcolor[rgb]{0.02,0.16,0.45}{##1}}}
\expandafter\def\csname PYG@tok@nv\endcsname{\def\PYG@tc##1{\textcolor[rgb]{0.73,0.38,0.84}{##1}}}
\expandafter\def\csname PYG@tok@s1\endcsname{\def\PYG@tc##1{\textcolor[rgb]{0.25,0.44,0.63}{##1}}}
\expandafter\def\csname PYG@tok@gp\endcsname{\let\PYG@bf=\textbf\def\PYG@tc##1{\textcolor[rgb]{0.78,0.36,0.04}{##1}}}
\expandafter\def\csname PYG@tok@sh\endcsname{\def\PYG@tc##1{\textcolor[rgb]{0.25,0.44,0.63}{##1}}}
\expandafter\def\csname PYG@tok@ow\endcsname{\let\PYG@bf=\textbf\def\PYG@tc##1{\textcolor[rgb]{0.00,0.44,0.13}{##1}}}
\expandafter\def\csname PYG@tok@sx\endcsname{\def\PYG@tc##1{\textcolor[rgb]{0.78,0.36,0.04}{##1}}}
\expandafter\def\csname PYG@tok@bp\endcsname{\def\PYG@tc##1{\textcolor[rgb]{0.00,0.44,0.13}{##1}}}
\expandafter\def\csname PYG@tok@c1\endcsname{\let\PYG@it=\textit\def\PYG@tc##1{\textcolor[rgb]{0.25,0.50,0.56}{##1}}}
\expandafter\def\csname PYG@tok@kc\endcsname{\let\PYG@bf=\textbf\def\PYG@tc##1{\textcolor[rgb]{0.00,0.44,0.13}{##1}}}
\expandafter\def\csname PYG@tok@c\endcsname{\let\PYG@it=\textit\def\PYG@tc##1{\textcolor[rgb]{0.25,0.50,0.56}{##1}}}
\expandafter\def\csname PYG@tok@mf\endcsname{\def\PYG@tc##1{\textcolor[rgb]{0.13,0.50,0.31}{##1}}}
\expandafter\def\csname PYG@tok@err\endcsname{\def\PYG@bc##1{\setlength{\fboxsep}{0pt}\fcolorbox[rgb]{1.00,0.00,0.00}{1,1,1}{\strut ##1}}}
\expandafter\def\csname PYG@tok@mb\endcsname{\def\PYG@tc##1{\textcolor[rgb]{0.13,0.50,0.31}{##1}}}
\expandafter\def\csname PYG@tok@ss\endcsname{\def\PYG@tc##1{\textcolor[rgb]{0.32,0.47,0.09}{##1}}}
\expandafter\def\csname PYG@tok@sr\endcsname{\def\PYG@tc##1{\textcolor[rgb]{0.14,0.33,0.53}{##1}}}
\expandafter\def\csname PYG@tok@mo\endcsname{\def\PYG@tc##1{\textcolor[rgb]{0.13,0.50,0.31}{##1}}}
\expandafter\def\csname PYG@tok@kd\endcsname{\let\PYG@bf=\textbf\def\PYG@tc##1{\textcolor[rgb]{0.00,0.44,0.13}{##1}}}
\expandafter\def\csname PYG@tok@mi\endcsname{\def\PYG@tc##1{\textcolor[rgb]{0.13,0.50,0.31}{##1}}}
\expandafter\def\csname PYG@tok@kn\endcsname{\let\PYG@bf=\textbf\def\PYG@tc##1{\textcolor[rgb]{0.00,0.44,0.13}{##1}}}
\expandafter\def\csname PYG@tok@o\endcsname{\def\PYG@tc##1{\textcolor[rgb]{0.40,0.40,0.40}{##1}}}
\expandafter\def\csname PYG@tok@kr\endcsname{\let\PYG@bf=\textbf\def\PYG@tc##1{\textcolor[rgb]{0.00,0.44,0.13}{##1}}}
\expandafter\def\csname PYG@tok@s\endcsname{\def\PYG@tc##1{\textcolor[rgb]{0.25,0.44,0.63}{##1}}}
\expandafter\def\csname PYG@tok@kp\endcsname{\def\PYG@tc##1{\textcolor[rgb]{0.00,0.44,0.13}{##1}}}
\expandafter\def\csname PYG@tok@w\endcsname{\def\PYG@tc##1{\textcolor[rgb]{0.73,0.73,0.73}{##1}}}
\expandafter\def\csname PYG@tok@kt\endcsname{\def\PYG@tc##1{\textcolor[rgb]{0.56,0.13,0.00}{##1}}}
\expandafter\def\csname PYG@tok@sc\endcsname{\def\PYG@tc##1{\textcolor[rgb]{0.25,0.44,0.63}{##1}}}
\expandafter\def\csname PYG@tok@sb\endcsname{\def\PYG@tc##1{\textcolor[rgb]{0.25,0.44,0.63}{##1}}}
\expandafter\def\csname PYG@tok@k\endcsname{\let\PYG@bf=\textbf\def\PYG@tc##1{\textcolor[rgb]{0.00,0.44,0.13}{##1}}}
\expandafter\def\csname PYG@tok@se\endcsname{\let\PYG@bf=\textbf\def\PYG@tc##1{\textcolor[rgb]{0.25,0.44,0.63}{##1}}}
\expandafter\def\csname PYG@tok@sd\endcsname{\let\PYG@it=\textit\def\PYG@tc##1{\textcolor[rgb]{0.25,0.44,0.63}{##1}}}

\def\PYGZbs{\char`\\}
\def\PYGZus{\char`\_}
\def\PYGZob{\char`\{}
\def\PYGZcb{\char`\}}
\def\PYGZca{\char`\^}
\def\PYGZam{\char`\&}
\def\PYGZlt{\char`\<}
\def\PYGZgt{\char`\>}
\def\PYGZsh{\char`\#}
\def\PYGZpc{\char`\%}
\def\PYGZdl{\char`\$}
\def\PYGZhy{\char`\-}
\def\PYGZsq{\char`\'}
\def\PYGZdq{\char`\"}
\def\PYGZti{\char`\~}
% for compatibility with earlier versions
\def\PYGZat{@}
\def\PYGZlb{[}
\def\PYGZrb{]}
\makeatother

\renewcommand\PYGZsq{\textquotesingle}

\begin{document}

\maketitle
\tableofcontents
\phantomsection\label{index::doc}


This library allows you to model transport through arbitrary control volumes using
rank StorAge Selection (rSAS) functions.


\chapter{Getting started}
\label{index:getting-started}\label{index:documentation-of-the-rsas-library}

\section{Before you install}
\label{index:before-you-install}
rsas depends on the Python libraries numpy, scipy, and cython, and the example
codes use pandas to wrangle the timeseries data. These must all be installed.
The Anaconda package (\href{https://store.continuum.io/cshop/anaconda/}{https://store.continuum.io/cshop/anaconda/}) contains all
the needed pieces, and is an easy way to get started. Install it before you
install rsas.


\section{Installation}
\label{index:installation}
The main part of the code is written in Cython to allow fast execution. Before
you use rsas you must comile and install it. Open a terminal in the rsas directory
and run:

\textgreater{} python setup.py install

It may take a few minutes. You may get warning messages, all of which can be ignored.
Error messages cannot though, and will prevent the compilation from completion.

Once the code has compiled successfully you don't need to do it again
unless this code is changed.


\section{Examples}
\label{index:examples}
Example uses of rsas are available:

examples/


\chapter{Further reading}
\label{index:further-reading}
The rSAS theory is described in:

Harman, C. J. (2014), Time-variable transit time distributions and transport:
Theory and application to storage-dependent transport of chloride in a watershed,
Water Resour. Res., 51, doi:10.1002/2014WR015707.


\section{Documentation for the code}
\label{index:documentation-for-the-code}\label{index:module-rsas}\index{rsas (module)}\index{solve() (in module rsas)}

\begin{fulllineitems}
\phantomsection\label{index:rsas.solve}\pysiglinewithargsret{\code{rsas.}\bfcode{solve}}{}{}
Solve the rSAS model for given fluxes
\begin{description}
\item[{Args: }] \leavevmode\begin{description}
\item[{J}] \leavevmode{[}n x 1 float64 ndarray{]}
Timestep-averaged inflow timeseries

\item[{Q}] \leavevmode{[}n x 2 float64 ndarray or list of n x 1 float64 ndarray{]}
Timestep-averaged outflow timeseries. Must have same units and length as J.
For multiple outflows, each column represents one outflow

\item[{rSAS\_fun}] \leavevmode{[}rSASFunctionClass or list of rSASFunctionClass generated by rsas.create\_function{]}
The number of rSASFunctionClass in this list must be the same as the 
number of columns in Q if Q is an ndarray, or elements in Q if it is a list.

\end{description}

\item[{Kwargs:}] \leavevmode\begin{description}
\item[{mode}] \leavevmode{[}`age' or `time' (default){]}
Numerical solution step order. `mode' refers to which variable is in the
outer loop of the numerical solution
\begin{description}
\item[{\code{mode='age'}}] \leavevmode
This is the original implementation used to generate the results in the paper.
It is slightly faster than the `time' implementation, but doesn't have the
memory-saving ``full\_outputs=False'' option. Only implemented for two
outfluxes, and there is no option to calculate 
output concentration timeseries inline. The calculated transit time
distributions must convolved with an imput concentration timeseries after the code has 
completed.

\item[{\code{mode='time'}}] \leavevmode
Slower, but easier to understand and build on than the `age' mode.
Memory savings come with the option to determine output concentrations
from a given input concentration progressively, and not retain the full
age-ranked storage and transit time distributions in memory (set
full\_outputs=False to take advantage of this).

\end{description}

\item[{ST\_init}] \leavevmode{[}m x 1 float64 ndarray{]}
Initial condition for the age-ranked storage. The length of ST\_init
determines the maximum age calculated. The first entry must be 0
(corresponding to zero age). To calculate transit time dsitributions up
to N timesteps in age, ST\_init should have length m = M + 1. The default
initial condition is ST\_init=np.zeros(len(J) + 1).

\item[{dt}] \leavevmode{[}float (default 1){]}
Timestep, assuming same units as J

\item[{n\_substeps}] \leavevmode{[}int (default 1){]}
(mode='age' only) If n\_substeps\textgreater{}1, the timesteps are subdivided to allow a more accurate
solution. Default is 1, which is also the value used in Harman (2015)

\item[{n\_iterations}] \leavevmode{[}int (default 3){]}
Number of iterations to converge on a consistent solution. Convergence 
in Harman (2015) was very fast, and n\_iterations=3 was adequate (also 
the default value here)

\item[{full\_outputs}] \leavevmode{[}bool (default True){]}
(mode='time' only) Option to return the full state variables array ST the cumulative
transit time distributions PQ1, PQ2, and other variables

\item[{C\_in}] \leavevmode{[}n x 1 float64 ndarray (default None){]}
(mode='time' only) Optional timeseries of inflow concentrations to convolved progressively
with the computed transit time distribution for the first flux in Q

\item[{C\_old}] \leavevmode{[}float (default None){]}
(mode='time' only) Optional concentration of the `unobserved fraction' of Q (from inflows 
prior to the start of the simulation) for correcting C\_out

\item[{evapoconcentration}] \leavevmode{[}bool (default False){]}
(mode='time' only) If True, it will be assumed that species in C\_in are not removed 
by the second flux, and instead become increasingly concentrated in
storage.

\end{description}

\item[{Returns:}] \leavevmode\begin{description}
\item[{A dict with the following keys:}] \leavevmode\begin{description}
\item[{`ST'}] \leavevmode{[}numpy float64 2D array{]}
Array of age-ranked storage for all ages and times. (full\_outputs=True only)

\item[{`PQ'}] \leavevmode{[}numpy float64 2D array{]}
List of time-varying cumulative transit time distributions. (full\_outputs=True only)

\item[{`Qout'}] \leavevmode{[}numpy float64 2D array{]}
List of age-based outflow timeseries. Useful for visualization. (full\_outputs=True only)

\item[{`theta'}] \leavevmode{[}numpy float64 2D array{]}
List of partial partition functions for each outflux. Keeps track of the
fraction of inputs that leave by each flux. This is needed to do
transport with evapoconcentration. (full\_outputs=True only)

\item[{`thetaS'}] \leavevmode{[}numpy float64 2D array{]}
Storage partial partition function fr each outflux. Keeps track of the
fraction of inputs that remain in storage. This is needed to do
transport with evapoconcentration. (full\_outputs=True only)

\item[{`MassBalance'}] \leavevmode{[}numpy float64 2D array{]}
Should always be within tolerances of zero, unless something is very
wrong. (full\_outputs=True only)

\item[{`C\_out'}] \leavevmode{[}list of numpy float64 1D array{]}
If C\_in is supplied, C\_out is the timeseries of outflow concentration 
in Q1. (mode='time' only)

\end{description}

\end{description}

\end{description}

For each of the arrays in the full outputs each row represents an age, and each
column is a timestep. For N timesteps and M ages, ST will have dimensions
(M+1) x (N+1), with the first row representing age T = 0 and the first
column derived from the initial condition.

\end{fulllineitems}

\index{create\_function() (in module rsas)}

\begin{fulllineitems}
\phantomsection\label{index:rsas.create_function}\pysiglinewithargsret{\code{rsas.}\bfcode{create\_function}}{}{}
Initialize an rSAS function
\begin{description}
\item[{Args:}] \leavevmode\begin{description}
\item[{rSAS\_type}] \leavevmode{[}str{]}
A string indicating the requested rSAS functional form.

\item[{params}] \leavevmode{[}n x k float64 ndarray{]}
Parameters for the rSAS function. The number of columns and 
their meaning depends on which rSAS type is chosen. For all the rSAS 
functions implemented so far, each row corresponds with a timestep.

\end{description}

\item[{Returns:}] \leavevmode\begin{description}
\item[{rSAS\_fun}] \leavevmode{[}rSASFunctionClass{]}
An rSAS function of the chosen type

\end{description}

\end{description}

The created function object will have methods that vary between types. All
must have a constructor (``\_\_init\_\_'') and two methods cdf\_all and cdf\_i. See
the documentation for rSASFunctionClass for more information.

Available choices for rSAS\_type, and a description of parameter array, are below.
These all take one parameter set (row) per timestep:
\begin{description}
\item[{`uniform'}] \leavevmode{[}Uniform distribution over the range {[}a, b{]}.{]}\begin{itemize}
\item {} 
\code{Q\_params{[}:, 0{]}} = a

\item {} 
\code{Q\_params{[}:, 1{]}} = b

\end{itemize}

\item[{`gamma': Gamma distribution}] \leavevmode\begin{itemize}
\item {} 
\code{Q\_params{[}:, 0{]}} = shift parameter

\item {} 
\code{Q\_params{[}:, 1{]}} = scale parameter

\item {} 
\code{Q\_params{[}:, 2{]}} = shape parameter

\end{itemize}

\item[{`gamma\_trunc'}] \leavevmode{[}Gamma distribution, truncated at a maximum value{]}\begin{itemize}
\item {} 
\code{Q\_params{[}:, 0{]}} = shift parameter

\item {} 
\code{Q\_params{[}:, 1{]}} = scale parameter

\item {} 
\code{Q\_params{[}:, 2{]}} = shape parameter

\item {} 
\code{Q\_params{[}:, 3{]}} = maximum value

\end{itemize}

\item[{`SS\_invgauss'}] \leavevmode{[}Produces analytical solution to the advection-dispersion equation{]}\begin{description}
\item[{(inverse Gaussian distribution) under steady-state flow.}] \leavevmode\begin{itemize}
\item {} 
\code{Q\_params{[}:, 0{]}} = scale parameter

\item {} 
\code{Q\_params{[}:, 1{]}} = Peclet number

\end{itemize}

\end{description}

\item[{`SS\_mobileimmobile'}] \leavevmode{[}Produces analytical solution to the advection-dispersion equation with{]}\begin{description}
\item[{linear mobile-immobile zone exchange under steady-state flow.}] \leavevmode\begin{itemize}
\item {} 
\code{Q\_params{[}:, 0{]}} = scale parameter

\item {} 
\code{Q\_params{[}:, 1{]}} = Peclet number

\item {} 
\code{Q\_params{[}:, 2{]}} = beta parameter

\end{itemize}

\end{description}

\item[{`from\_steady\_state\_TTD'}] \leavevmode{[}rSAS function constructed to reproduce a given transit time distribution{]}\begin{description}
\item[{at steady flow. Assumes timestep is dt=1.}] \leavevmode\begin{itemize}
\item {} 
\code{Q\_params{[}0, 0{]}} = Steady flow rate

\item {} 
\code{Q\_params{[}1:, 0{]}} = Steady-state transit time distribution, from T = dt.

\end{itemize}

\end{description}

\end{description}

\end{fulllineitems}

\index{transport() (in module rsas)}

\begin{fulllineitems}
\phantomsection\label{index:rsas.transport}\pysiglinewithargsret{\code{rsas.}\bfcode{transport}}{\emph{PQ}, \emph{C\_in}, \emph{C\_old}}{}
Apply a time-varying transit time distribution to an input concentration timseries
\begin{description}
\item[{Args:}] \leavevmode\begin{description}
\item[{PQ}] \leavevmode{[}numpy float64 2D array, size N x N{]}
The CDF of the backwards transit time distribution P\_Q1(T,t)

\item[{C\_in}] \leavevmode{[}numpy float64 1D array, length N.{]}
Timestep-averaged inflow concentration.

\item[{C\_old}] \leavevmode{[}numpy float64 1D array, length N.{]}
Concentration to be assumed for portion of outflows older than initial
timestep

\end{description}

\item[{Returns:}] \leavevmode\begin{description}
\item[{C\_out}] \leavevmode{[}numpy float64 1D array, length N.{]}
Timestep-averaged outflow concentration.

\item[{C\_mod\_raw}] \leavevmode{[}numpy float64 1D array, length N.{]}
Timestep-averaged outflow concentration, prior to correction with C\_old.

\item[{observed\_fraction}] \leavevmode{[}numpy float64 1D array, length N.{]}
Fraction of outflow older than the first timestep

\end{description}

\end{description}

\end{fulllineitems}

\index{transport\_with\_evapoconcentration() (in module rsas)}

\begin{fulllineitems}
\phantomsection\label{index:rsas.transport_with_evapoconcentration}\pysiglinewithargsret{\code{rsas.}\bfcode{transport\_with\_evapoconcentration}}{\emph{PQ}, \emph{thetaQ}, \emph{thetaS}, \emph{C\_in}, \emph{C\_old}}{}
Apply a time-varying transit time distribution to an input concentration timseries
\begin{description}
\item[{Args:}] \leavevmode\begin{description}
\item[{PQ}] \leavevmode{[}numpy float64 2D array, size N+1 x N+1{]}
The CDF of the backwards transit time distribution P\_Q1(T,t)

\item[{thetaQ, thetaS}] \leavevmode{[}numpy float64 2D array, size N+1 x N+1{]}
Partial partition functions for discharge and storage

\item[{C\_in}] \leavevmode{[}numpy float64 1D array, length N.{]}
Timestep-averaged inflow concentration.

\item[{C\_old}] \leavevmode{[}numpy float64 1D array, length N.{]}
Concentration to be assumed for portion of outflows older than initial
timestep

\end{description}

\item[{Returns:}] \leavevmode\begin{description}
\item[{C\_out}] \leavevmode{[}numpy float64 1D array, length N.{]}
Timestep-averaged outflow concentration.

\item[{C\_mod\_raw}] \leavevmode{[}numpy float64 1D array, length N.{]}
Timestep-averaged outflow concentration, prior to correction with C\_old.

\item[{observed\_fraction}] \leavevmode{[}numpy float64 1D array, length N.{]}
Fraction of outflow older than the first timestep

\end{description}

\end{description}

\end{fulllineitems}



\chapter{Indices and tables}
\label{index:indices-and-tables}\begin{itemize}
\item {} 
\emph{genindex}

\item {} 
\emph{modindex}

\item {} 
\emph{search}

\end{itemize}


\renewcommand{\indexname}{Python Module Index}
\begin{theindex}
\def\bigletter#1{{\Large\sffamily#1}\nopagebreak\vspace{1mm}}
\bigletter{r}
\item {\texttt{rsas}}, \pageref{index:module-rsas}
\end{theindex}

\renewcommand{\indexname}{Index}
\printindex
\end{document}
